The running time of a function written in the programming language Python depends on the exact technique used to implement the desired functionality. Some tasks are desired to be especially fast, such as sorting algorithms. One example of such a sorting algorithm is the insertion sort, which will be used and analyzed in this project. 
It is often difficult to know which techniques are the most efficient ones, and the answers are sometimes unexpected. Often you have to choose between a more readable implementation, and something which you might suspect to be more efficient. Where should the compromise between readability and efficiency lie? Sometimes you know that some choice must be faster than the alternative, but to what degree it is more efficent might be unknown. This is the reason why we have chosen to investigate the effect of how different Python programming techniques affect the runtime of a sample algorithm implementation, in our case the insertion sort.
It will be nice to gain some more insight into these questions, and therefore is this our focus for this project.
