The running time of a function written in the programming language Python depends on the exact technique used to implement the desired functionality. Some tasks are desired to be especially fast, such as sorting algorithms. One example of such a sorting algorithm is the insertion sort, which can be implemented in several different ways in Python. 
It is often difficult to know which techniques are the most efficient ones, and the answers are sometimes surprising. Often you have to choose between a more readable implementation, or something which you suspect to be more efficient. This is the reason why we have chosen to investigate the effect of how different Python programming techniques affect the runtime of a sample algorithm implementation, in our case the insertion sort.

%Where should the compromise between readability and efficiency lie?
