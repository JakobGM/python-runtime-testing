The insertion sort algorithm is conceptually quite simple. When implementing it, the programmer does not have a lot of choices, except for how to exactly use the programming language and its features. We have chosen to investigate the effect of 4 such factors on the runtime of the algorithm.    

\begin{itemize}
    \item Factor A: How to implement an interchange of values between two variables. Here you have the choice between creating a temporary variable in order to store the value of the first variable, while you change it to the value of the second one.
    \begin{lstlisting}[language=Python]
        temp = array[j-1]
        array[j-1] = array[j]
        array[j] = temp
    \end{lstlisting}
    Versus using the more idiomatic aproach (often called `pythonic') of tuple reassignment, as shown:
    \begin{lstlisting}[language=Python]
        array[j-1], array[j] = array[j], array[j-1]
    \end{lstlisting}
 
    \item Factor B: How to create an iteration variable in order to keep track of how much of the array that has already been sorted. There are two ways  
\end{itemize}
